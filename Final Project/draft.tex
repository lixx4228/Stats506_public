% Options for packages loaded elsewhere
\PassOptionsToPackage{unicode}{hyperref}
\PassOptionsToPackage{hyphens}{url}
%
\documentclass[
]{article}
\usepackage{lmodern}
\usepackage{amssymb,amsmath}
\usepackage{ifxetex,ifluatex}
\ifnum 0\ifxetex 1\fi\ifluatex 1\fi=0 % if pdftex
  \usepackage[T1]{fontenc}
  \usepackage[utf8]{inputenc}
  \usepackage{textcomp} % provide euro and other symbols
\else % if luatex or xetex
  \usepackage{unicode-math}
  \defaultfontfeatures{Scale=MatchLowercase}
  \defaultfontfeatures[\rmfamily]{Ligatures=TeX,Scale=1}
\fi
% Use upquote if available, for straight quotes in verbatim environments
\IfFileExists{upquote.sty}{\usepackage{upquote}}{}
\IfFileExists{microtype.sty}{% use microtype if available
  \usepackage[]{microtype}
  \UseMicrotypeSet[protrusion]{basicmath} % disable protrusion for tt fonts
}{}
\makeatletter
\@ifundefined{KOMAClassName}{% if non-KOMA class
  \IfFileExists{parskip.sty}{%
    \usepackage{parskip}
  }{% else
    \setlength{\parindent}{0pt}
    \setlength{\parskip}{6pt plus 2pt minus 1pt}}
}{% if KOMA class
  \KOMAoptions{parskip=half}}
\makeatother
\usepackage{xcolor}
\IfFileExists{xurl.sty}{\usepackage{xurl}}{} % add URL line breaks if available
\IfFileExists{bookmark.sty}{\usepackage{bookmark}}{\usepackage{hyperref}}
\hypersetup{
  pdftitle={STATS 506 Final Project},
  pdfauthor={Chuwen Li},
  hidelinks,
  pdfcreator={LaTeX via pandoc}}
\urlstyle{same} % disable monospaced font for URLs
\usepackage[margin=1in]{geometry}
\usepackage{graphicx}
\makeatletter
\def\maxwidth{\ifdim\Gin@nat@width>\linewidth\linewidth\else\Gin@nat@width\fi}
\def\maxheight{\ifdim\Gin@nat@height>\textheight\textheight\else\Gin@nat@height\fi}
\makeatother
% Scale images if necessary, so that they will not overflow the page
% margins by default, and it is still possible to overwrite the defaults
% using explicit options in \includegraphics[width, height, ...]{}
\setkeys{Gin}{width=\maxwidth,height=\maxheight,keepaspectratio}
% Set default figure placement to htbp
\makeatletter
\def\fps@figure{htbp}
\makeatother
\setlength{\emergencystretch}{3em} % prevent overfull lines
\providecommand{\tightlist}{%
  \setlength{\itemsep}{0pt}\setlength{\parskip}{0pt}}
\setcounter{secnumdepth}{-\maxdimen} % remove section numbering
\ifluatex
  \usepackage{selnolig}  % disable illegal ligatures
\fi

\title{STATS 506 Final Project}
\author{Chuwen Li}
\date{12/15/2020}

\begin{document}
\maketitle

\hypertarget{introduction}{%
\section{\texorpdfstring{\protect\hypertarget{jump}{}{Introduction}}{Introduction}}\label{introduction}}

Nowadays there is a growing trend in formula feeding as the nutritional
composition of infant formula is increasingly close to that of breast
milk and more young mothers get depressed from breastfeeding pain.
Still, lots of parents are worried or blamed by their parents for
risking infants' health by only choose formula feeding. There is no
doubt that many merits associated with breastfeeding, including reducing
risk of asthma and infections. Some of which are scientifically proved
while others are widely various in opinions. This project aims to answer
one of the widespread breastfeeding questions:

\textbf{Does breastfeeding prevent early childhood overweight?}

The analysis is based on fitting linear models with body measure index
(BMI) as the dependent variable, and the explanatory factors as
covariates, including sex, ethnicity, ever breastfed, mother smoked
during pregnancy. The target group is 3 to 6 years old children.

\hypertarget{datamethods}{%
\section{Data/Methods}\label{datamethods}}

The data used is from the National Health and Nutrition Examination
Survey
(\href{https://wwwn.cdc.gov/nchs/nhanes/continuousnhanes/default.aspx?BeginYear=2017}{NHANES}).
Specific data sets used are: Demographic data on the participants,
health and nutrition examination survey on the participants, the answers
to a diet behavior questionnaire, the answers to a early childhood
questionnaire.

The analysis was performed using the ``survey'' package in R, which
involves primary sampling unit, strata, and survey weights adjustment.
The code for the analysis is available
\href{https://github.com/lixx4228/Stats506_public}{here}. Two types of
linear models were considered. The first one was taking the predict
variable as a continuous outcome. The second model used a newly created
binary variable: `overweight' as the predictor, which took BMI greater
and equal to 90th percentile as 1, and 0 otherwise. Both models were
fitted with age, sex, ethnicity, mother ever smoked during pregnancy,
and ever being breastfed. Figure 1 shows simple means for each
explanatory variable with BMI.

\begin{figure}
\centering
\includegraphics{draft_files/figure-latex/figs-1.pdf}
\caption{\textbf{Figure 1.} \emph{BMI against age, sex, ethnicity, and
momther smoked during pregnancy between 3-to-6-year-old breastfed and
non-breastfed children, 2017-2018.}}
\end{figure}

\hypertarget{results}{%
\section{Results}\label{results}}

Table 1 displays the summary results for logistic regression model and
OLS model. For the OLS model, a natural spline with 2 knots was applied
to the `Birth weight' variable. Both models showed a negative
association between breastfed children and being overweight or BMI gains
with those younger than 6, of which parameters are statistically
significant for 95\% confidence level. There was a reduced risk of being
overweight for breastfed children and based on the odds ratio, the risk
was around 0.5 times smaller (95\% CI, 0.37 - 0.75). The OLS model gave
a result of 0.7 less BMI value for an ever breastfed child compared to a
never breastfed child, holding all other variables constant. For other
variables, age was positively associated with overweight in the logistic
model but it was not significant in the OLS model. In the data, no Asian
ethnicity children whose BMI exceeded 90 percentile.

Breastfed children whose mothers never smoked during pregnancy were
associated with a lower risk of overweight (OR 0.387) and the risk
difference is almost doubled. Children in sex, age, and ethnicity subset
behaves the same as the whole group.

\hypertarget{conclusion}{%
\section{Conclusion}\label{conclusion}}

National data obtained from children (3-6 years) in 2017 and 2018 show
ever breastfed children are less likely to experience early childhood
overweight (BMI exceeds 90 percentile) compared to children who are only
fed with formula. This result, however, does not entirely consistent
with previous overweight and breastfeeding association study on
1998-1994 NHANES data (Hediger et al., 2001). Hediger, et. al.~used BMI
exceeded 95 percentile as the indicator of overweight and found no
reduced risk of being overweight between breastfed and non-breastfed
children. Due to the technical difficulty of combining survey weights
and mixed models in R, my study is limited to simple linear models.
Thus, future application of mixed model to this data set is a field of
interest.

\textbf{Table 1.} \emph{Statistical summary of logistic model and OLS
model.}

~

Overweight(BMI ≥ 90th Percentile)

BMI(kg/m²)

Predictors

Odds Ratios

CI

p

Estimates

CI

p

(Intercept)

0.019

0.003~--~0.103

0.006

16.000

14.561~--~17.438

\textless0.001

Breastfed{[}Yes{]}

0.523

0.366~--~0.748

0.016

-0.696

-1.050~--~-0.341

0.018

Birth weight, pounds

1.170

0.945~--~1.448

0.209

Age, years

1.482

1.204~--~1.823

0.014

0.048

-0.125~--~0.221

0.616

Sex{[}Male{]}

1.245

0.585~--~2.650

0.595

0.258

-0.203~--~0.720

0.334

Ethnicity{[}Hispanic{]}

1.283

0.313~--~5.262

0.743

0.422

-0.874~--~1.717

0.558

Ethnicity{[}White{]}

0.321

0.128~--~0.810

0.061

-0.593

-1.343~--~0.158

0.196

Ethnicity{[}AfricanAmerican{]}

0.285

0.113~--~0.724

0.046

-0.576

-1.311~--~0.159

0.199

Ethnicity{[}Asian{]}

0.000

0.000~--~0.000

\textless0.001

-1.514

-2.484~--~-0.544

0.038

Ethnicity{[}Other{]}

0.601

0.231~--~1.559

0.343

-0.569

-1.475~--~0.337

0.286

Mother smoked when pregnant{[}Yes{]}

1.121

0.629~--~1.997

0.715

0.189

-0.319~--~0.697

0.507

ns(Birth weight, 2){[}1{]}

2.982

1.587~--~4.377

0.014

ns(Birth weight, 2){[}2{]}

1.541

0.341~--~2.741

0.066

Observations

625

625

R2 / R2 adjusted

0.102 / -111.048

0.078 / -142.773

\begin{center}\rule{0.5\linewidth}{0.5pt}\end{center}

\hypertarget{reference}{%
\section{Reference}\label{reference}}

{[}1{]} Hediger ML, Overpeck MD, Kuczmarski RJ, Ruan WJ. Association
Between Infant Breastfeeding and Overweight in Young Children.
\emph{JAMA}. 2001;285(19):2453--2460. \url{doi:10.1001/jama.285.19.2453}

\end{document}
